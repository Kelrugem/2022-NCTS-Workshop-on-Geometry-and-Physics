\documentclass[hyperref={pdfpagelabels=false}]{beamer}
\usepackage{CJKutf8}
\usepackage[english]{babel}
\usepackage{xcolor}
\usepackage{lmodern}
\usepackage{amssymb}
\usepackage[makeroom]{cancel} %for crossing symbols
%\usepackage{calligra}
%\DeclareMathAlphabet{\mathcalligra}{T1}{calligra}{m}{n} %For small \mathcal letters
\makeatletter
\DeclareFontEncoding{LS1}{}{}
\DeclareFontSubstitution{LS1}{stix}{m}{n}
\DeclareMathAlphabet{\mathKel}{LS1}{stixscr}{m}{n}
\DeclareMathAlphabet{\mathcal}{LS1}{stixscr}{m}{n}
\usepackage{amsthm}
\usepackage{amsmath}
%\usepackage{mathabx}
\usepackage{stmaryrd}
\usepackage{amsbsy}
\usepackage{dsfont}
\usepackage{mathtools} %für mathclap und coloneqq
%\usepackage{amsbsy}
\usepackage{mleftright} %Distanz zu \left \right weg
\usepackage{tikz-cd}

\usepackage{tabularx} %Automatic line break of tables using X instead c l r
%\usepackage{longtable} %table auf mehreren Seiten
%\usepackage{ltxtable} %Combination of both above
\usepackage{xcolor, colortbl}

%Für die ganzen Diagramme
\usepackage{pgfplots}
\usepackage{graphicx} %Für raisebox, vertical displacement of figures
\usetikzlibrary{decorations.markings}

\definecolor{Gray}{gray}{0.85}
%\usepackage[style=authortitle-icomp]{biblatex}
%\usepackage[babel,german=guillemets]{csquotes}

%\setcounter{tocdepth}{4}
%\setcounter{tocdepth}{5}
%\setcounter{secnumdepth}{4}
%\setcounter{secnumdepth}{5}
\usepackage[backend=biber, style=numeric]{biblatex}
\addbibresource{Literatur.bib}
\newcommand{\footlineextra}[1]{
    \begin{tikzpicture}[remember picture,overlay]
        \node[yshift=2ex,anchor=south west] at (current page.south west) {\usebeamerfont{author in head/foot}\hspace{2ex}#1};
    \end{tikzpicture}
}
%\DeclareMathOperator{\sAut}{\mathKel{A\mkern-5.5mu u\mkern-4mu t\mkern-1.5mu}}
%\DeclareMathOperator{\saut}{\mathKel{a\mkern-4.5mu u\mkern-4mu t\mkern-1.5mu}}
%\DeclareMathOperator{\sEnd}{\mathKel{E\mkern-4mu n\mkern-4.5mu d\mkern-1mu}}
%\setbeamertemplate{bibliography entry title}{}
%\setbeamertemplate{bibliography entry location}{}
%\setbeamertemplate{bibliography entry note}{}
%\setbeamertemplate{bibliography item}{\insertbiblabel}

%\pagestyle{headings}
%
%\makeatletter
%\patchcmd{\beamer@calculateheadfoot}{\advance\footheight by 5pt}{\advance\footheight by 20pt}{}{}
%\makeatother

%\makeatletter
%\setbeamertemplate{footline}
%{
  %\leavevmode%
  %\hbox{%
  %\begin{beamercolorbox}[wd=.175\paperwidth,ht=2.25ex,dp=1ex,center]{author in head/foot}%
    %\usebeamerfont{author in head/foot}\insertshortauthor\expandafter\beamer@ifempty\expandafter{\beamer@shortinstitute}{}{~~(\insertshortinstitute)}
  %\end{beamercolorbox}%
  %\begin{beamercolorbox}[wd=.65\paperwidth,ht=2.25ex,dp=1ex,center]{title in head/foot}%
    %\usebeamerfont{title in head/foot}\insertshorttitle
  %\end{beamercolorbox}%
  %\begin{beamercolorbox}[wd=.175\paperwidth,ht=2.25ex,dp=1ex,right]{date in head/foot}%
    %\usebeamerfont{date in head/foot}\insertshortdate{}\hspace*{2em}
    %%\insertframenumber{} 
		%%/ \inserttotalframenumber\hspace*{2ex} 
  %\end{beamercolorbox}}%
  %\vskip0pt%
%}
%\makeatother 

%\setbeamercolor{footlinecolor}{fg=white,bg=Navyblue}
\newcommand\insertreferences{}
\setbeamertemplate{footline}{%
  \leavevmode%
  \hbox{%
  \begin{beamercolorbox}[wd=.09\paperwidth, ht=5ex,dp=1ex,center, sep=1.4ex]{author in head/foot}%
    \usebeamerfont{author in head/foot}
		%\vfill
		%Sources
		%\vfill
		Sources
  \end{beamercolorbox}%
  \begin{beamercolorbox}[wd=.91\paperwidth,ht=5ex,dp=1ex,center]{title in head/foot}%
    \usebeamerfont{title in head/foot}
    \insertreferences

  \end{beamercolorbox}
}
}

%\setbeamertemplate{footline}
%{
  %\leavevmode%
  %\hbox{%
  %\begin{beamercolorbox}[wd=.09\paperwidth,ht=4.35ex,dp=20ex,center]{author in head/foot}%
    %\usebeamerfont{author in head/foot}
		%Source
  %\end{beamercolorbox}%
  %\begin{beamercolorbox}[wd=.91\paperwidth,ht=4.35ex,dp=20ex,center]{title in head/foot}%
    %\usebeamerfont{title in head/foot}
		%Camilo Arias Abad, Marius Crainic. Representations up to homotopy of Lie algebroids. \textit{Journal für die reine und angewandte Mathematik (Crelles Journal)}, 2012(663):91–126, 2012.
  %\end{beamercolorbox}%
  %%\begin{beamercolorbox}[wd=.175\paperwidth,ht=2.25ex,dp=1ex,right]{date in head/foot}%
    %%\usebeamerfont{date in head/foot}\insertshortdate{}\hspace*{2em}
    %%%\insertframenumber{} 
		%%%/ \inserttotalframenumber\hspace*{2ex} 
  %%\end{beamercolorbox}
	%}%
  %\vskip0pt%
%}

\title{Curved Yang-Mills gauge theories}   
\subtitle{based on my preprint arXiv:2210.02924}   
\author{Simon-Raphael Fischer} 
\institute{\begin{CJK*}{UTF8}{bkai}國家理論科學研究中心\end{CJK*}}
\date{30 December 2022} 
%\date{Le lundi 31 mai 2021} 

% zusaetzlich ist das usepackage{beamerthemeshadow} eingebunden 
%\usepackage{beamerthemeIlmenau}
%\usepackage{beamerthemeshadow}
\usepackage{beamerthemeDarmstadt}

%  \beamersetuncovermixins{\opaqueness<1>{25}}{\opaqueness<2->{15}}
%  sorgt dafuer das die Elemente die erst noch (zukuenftig) kommen 
%  nur schwach angedeutet erscheinen 
\beamersetuncovermixins{\opaqueness<1>{25}}{\opaqueness<2->{15}}
% klappt auch bei Tabellen, wenn teTeX verwendent wird\ldots

\beamertemplatenavigationsymbolsempty %Damit sind die kleinen Navigationssymbole unten weg

%\usesectionheadtemplate{}{}
%\usesubsectionheadtemplate{}{}

\def\be{\begin{equation}}
\def\ee{\end{equation}}
\def\bs{\begin{subequations}}
\def\es{\end{subequations}}
\def\ba#1\ea{\begin{align}#1\end{align}}
\def\bes{\begin{equation*}}
\def\ees{\end{equation*}}
\def\bas#1\eas{\begin{align*}#1\end{align*}}

\AtBeginEnvironment{remark}{%
  \setbeamercolor{block title}{use=example text,fg=black,bg=yellow!75!black}
  \setbeamercolor{block body}{parent=normal text,use=block title example,bg=yellow!10}
}

\renewcommand{\qedsymbol}{}
\theoremstyle{plain}
\newtheorem{conjecture}[theorem]{Conjecture}
\newtheorem{proposition}[theorem]{Proposition}
%\newtheorem{definition}[theorem]{Definition}
\theoremstyle{remark}
\newtheorem*{remark}{Remarks}
\newtheorem*{idea}{Idea}
\newtheorem*{motivation}{Motivation}
\newtheorem*{summary}{Summary}
\newtheorem*{situation}{Situation}
\newtheorem*{lab}{Situation: Lie algebra bundles}
\newtheorem*{question}{Question}
\newtheorem*{fieldredefinition}{Field Redefinition}
\newtheorem*{construction}{Construction}
\newtheorem*{aim}{Aim}
\newtheorem*{BackToTheRoots}{Guide to recover the classical theory}

\AtBeginEnvironment{BackToTheRoots}{%
  \setbeamercolor{block title}{use=example text,fg=black,bg=pink!75!black}
  \setbeamercolor{block body}{parent=normal text,use=block title example,bg=pink!10}
}

%\theoremstyle{definition}
%\newtheorem{definition}[theorem]{Definition}
%\newtheorem*{SecondIn}{Second Inequality}

%mathrm mit mathup ersetzen, damit die font passt
\renewcommand\familydefault{\sfdefault} %comment to see the difference
\DeclareMathAlphabet      {\mathup}{OT1}{\familydefault}{m}{n}


\begin{document}


\begin{frame}
\thispagestyle{empty}
\titlepage
\end{frame} 


{
\setbeamertemplate{footline}{}
\begin{frame}
\frametitle{Table of contents}
\tableofcontents
\end{frame} 
}
\section{Infinitesimal theory}
%\subsection{Motivation and short introduction}
{
\setbeamertemplate{footline}{}
\begin{frame}{\textbf{Infinitesimal} curved Yang-Mills-Higgs gauge theory}
\centering
\begin{tikzpicture}
\filldraw [fill=gray!10!white, draw=black] plot [smooth cycle] coordinates {(5,0.25) (6,0.35) (6.5, 0.2) (7,0.5) (7,1.65) (6.5,2.75) (5.8,2.75) (5.3,1.45) (4.8,0.85) } node at (6.2,1.7) {$(N, g)$} node at (6.2, 3.2) {Riemannian manifold};
\path[->] (2.6,1.7) edge [bend left] node[above] {Higgs field} node[below] {$\Psi$} (5.0,1.7);

\filldraw [fill=gray!10!white, draw=black] (-0.2,.5) to[bend left] (1.3,2.85) to[bend left] (2.7,2) to[bend right] (1.8,.5) to[bend right] cycle node at (1.2,1.7) {$(M, \eta)$} node at (1.2, 3.2) {Spacetime};
\path[->] (0.7,0.5) edge [bend right] node[left] {$A \in \Omega^1$} node[right] {Gauge bosons} (0.7, -1.5);
\filldraw [fill=gray, draw=black] (5.2, .7) circle (0.2);
\path[->] (2.5,-1.5) edge [bend right] node[right] {action $\gamma$} (5.2, .7);
\draw (-0.2,-1.7)-- (1.25, -1.7) node[below] {Lie algebra $\mathfrak{g}$} --(2.7,-1.7);

\tikzset{shift={(5,-3)}} %Change position of the next drawing
\begin{axis}[ scale = 0.6,
            hide axis,
            %axis lines=middle,
%            axis on top,
%            axis line style={blue,dashed,thick},
%            ymin=-2,ymax=2,
%            xmin=-2,xmax=2,
%            zmin=-2,zmax=2,
            samples=40,
            domain=0:360,
            y domain=0:1.25,clip=false
        ]
        \addplot3 [surf, shader=flat, draw=black, fill=gray!10!white, z buffer=sort]
           ({sin(x)*y}, {cos(x)*y}, {(y^2-1)^2});
        \draw[blue,thick,dashed] (axis cs:0,0,0) -- (axis cs:1,0,0)
                    node[below,font=\footnotesize]{};
        \draw[blue,thick,-stealth] (axis cs:1,0,0) -- (axis cs:1.3,0,0)
                    node[above,font=\footnotesize]{};
        \draw[blue,thick,dashed] (axis cs:0,0,0) -- (axis cs:0,-1,0)
                    node[left=2mm,font=\footnotesize]{}; %{Label} am Ende 
        \draw[blue,thick,-stealth] (axis cs:0,-1,0) -- (axis cs:0,-1.5,0)
                    node[right=1mm,font=\footnotesize]{};
        \draw[blue,thick,dashed] (axis cs:0,0,0) -- (axis cs:0,0,1)
                    %node[left=2mm,font=\footnotesize]{$\phi_{\text{RE}}$}
                    ;
        \draw[blue,thick,-stealth] (axis cs:0,0,1) -- (axis cs:0,0,1.3)
                    node[right,font=\footnotesize]{$\mathcal{V}$};
        \end{axis}
\end{tikzpicture}

\end{frame}

\begin{frame}{Guide: Infinitesimal curved Yang-Mills-Higgs gauge theory}
\setbeamercovered{invisible}
\begin{table}[h!]
		\begin{tabularx}{\textwidth}{X X}
			\rowcolor{gray}
			Classical formalism & CYMH GT \\
			Lie algebra $\mathfrak{g}$ as $M \times \mathfrak{g}$ & Lie algebroid $E \to N$ \\
			\rowcolor{Gray}
			$\mathfrak{g}$-action $\gamma$ & Anchor $\rho$ of $E$ \\ 
			\rowcolor{Gray}
			& \& $E$-connections \\
			Canonical flat connection $\nabla^0$ on $M \times \mathfrak{g}$ & General connection $\nabla$ on $E$
		\end{tabularx}
\end{table}
\pause
\begin{remark}[Why a "curved theory"?]
Usually, the field strength $F$ is given by (abelian, for simplicity)
\bas
F
&\coloneqq
\mathrm{d}A
=
\mathrm{d}^{\nabla^0}A.
\eas
$\rightsquigarrow$ We will use a general connection $\nabla$ instead of $\nabla^0$, and $\nabla$ may not be flat.
\end{remark}
\end{frame}
}

\section{Integration: Ansatz}
\subsection{Principal bundles based on Lie group bundle actions}
{
\setbeamertemplate{footline}{}
\begin{frame}
\textbf{We will only focus on Yang-Mills theories:}

\begin{table}[h!]
		\begin{tabularx}{\textwidth}{X| c c} 
			\rowcolor{gray}
			& Classical & Curved \\ \hline
			Infinitesimal & Lie algebra $\mathfrak{g}$ & LAB\footnote{LAB = Lie algebra bundle} $\mathcal{g}$ \\
			\rowcolor{Gray}
			Integrated & Lie group $G$ & \textcolor[rgb]{1,0.41,0.13}{LGB\footnote{LGB = Lie group bundle} $\mathcal{G}$} \\
		\end{tabularx}
\end{table}

\begin{center}
	\begin{tikzcd}[ampersand replacement=\&]
	G \arrow{r} \& \mathcal{G} \arrow{d} \\
	\& M
	\end{tikzcd}
\end{center}

\end{frame}
}

\renewcommand\insertreferences{{\tiny  K. Mackenzie. General Theory of Lie Groupoids and Algebroids. \newline \textit{London Mathematical Society Lecture Note Series}, 213, 2005.}}

\begin{frame}
\begin{definition}[LGB actions, simplified]
\begin{center}
	\begin{tikzcd}[ampersand replacement=\&, column sep = small, row sep = small]
	\& \mathcal{G} \arrow{d} \\
	\mathcal{P} \arrow{r}{\pi} \& M
	\end{tikzcd}
\end{center}
$\mathcal{P} \stackrel{\pi}{\to} M$ a fibre bundle. A \textbf{right-action of $\mathcal{G}$ on $\mathcal{P}$} is a smooth map 
%\bas
$\mathcal{P} * \mathcal{G} \coloneqq \pi^*\mathcal{G} = \mathcal{P} \times_M \mathcal{G} \to \mathcal{P}$,
$(p, g) \mapsto p \cdot g$,
%\eas
satisfying the following properties:
\ba\label{InvarianceOffUnderGAction}
\pi(p \cdot g) &= \pi(p),\\
(p \cdot g) \cdot h &= p \cdot (gh),\\
p \cdot e_{\pi(p)} &= p
\ea
for all $p \in \mathcal{P}$ and $g, h \in \mathcal{G}_{\pi(p)}$, where $e_{\pi(p)}$ is the neutral element of $\mathcal{G}_{\pi(p)}$.
\end{definition}
\end{frame}

{
\setbeamertemplate{footline}{}
\begin{frame}{Examples}
	\begin{example}
	$\mathcal{G}$ acts canonically on itself:
	\bas
	\mathcal{G} * \mathcal{G} &\to \mathcal{G},\\
	(q,h) &\mapsto qh.
	\eas
	\end{example}
\pause
\begin{example}[Recovering Lie group action]
$\bullet$ Either by $M = \{*\}$.

$\bullet$ Or by $\mathcal{G} \cong M \times G$, then also $\mathcal{P} * \mathcal{G} \cong \mathcal{P} \times G$, and we can define
\bas
\mathcal{P} \times G &\to \mathcal{P}, \\
(p, g) &\mapsto p \cdot g \coloneqq p \cdot \bigl( \pi(p), g \bigr),
\eas
which is equivalent to $\mathcal{P} * \mathcal{G} \to \mathcal{P}$.
\end{example}
\end{frame}

\begin{frame}
\begin{BackToTheRoots}[Part 1 of 3]
\begin{enumerate}
	\item $\mathcal{G} \cong M \times G$
\end{enumerate}
\end{BackToTheRoots}
\end{frame}
}

\renewcommand\insertreferences{{\tiny Ieke Moerdijk, Janez Mrcun. Introduction to Foliations and Lie Groupoids. \newline \textit{Cambridge Studies in Advanced Mathematics 91, Cambridge University Press, Cambridge}, 2003}}

\begin{frame}
\begin{definition}[Principal bundle]
Still a fibre bundle
\begin{center}
	\begin{tikzcd}[ampersand replacement=\&]
	G \arrow{r} \& \mathcal{P} \arrow{d}{\pi} \\
	\& M
	\end{tikzcd}
\end{center}
but with $\mathcal{G}$-action
\bas
\begin{matrix}
	\textcolor[rgb]{1,0,0}{\xcancel{\mathcal{P} \times G}} &\to \mathcal{P} \\
	\mathcal{P} * \mathcal{G} &
\end{matrix}
\eas
simply transitive on fibres of $\mathcal{P}$, and "suitable" atlas.
\end{definition}
\end{frame}
\subsection{Connections as parallel transport}
{
\setbeamertemplate{footline}{}
\begin{frame}{Connection on $\mathcal{P}$: Idea}
\begin{figure}
%\caption{Pushforwards via right-multiplication of tangent vectors}  
%\label{figure:fibre bundle}
\centering
\begin{tikzpicture}[scale=0.55]
\draw (0,-3.5) to[out=10,in=170] (4.25,-3.5) to[out=350,in=190] (8.5,-3.5);
\draw (1.25,2.5) to[out=280,in=90](1.5,0) to[out=270,in=80] (1.25,-2.5);
\filldraw [fill=gray!20!white, draw=white] (3.5,2.5) to[out=280,in=90](3.75,0) to[out=270,in=80] (3.5,-2.5) -- (5,-2.5)  to[out=80,in=270](5.25,0) to[out=90,in=280] (5,2.5) -- cycle;
\draw (4.25,2.5) to[out=280,in=95](4.41,1) to[out=275,in=85] (4.41,-1) to[out=265,in=80] (4.25,-2.5); %draw with middle section for precise point placement

\draw (4,-1) node {\textcolor[rgb]{1,0,0}{$p$}};
\draw [<-, draw=blue] (5,1) to[out=275,in=85] (5,-1);
\draw (5.5,0) node {\textcolor[rgb]{0,0,1}{$\cdot g$}};
\filldraw [fill=red, draw=red] (4.41,1) circle (1pt);
\filldraw [fill=red, draw=red] (4.41,-1) circle (1pt);
\draw (5.75,2.5) to[out=280,in=90](6,0) to[out=270,in=80] (5.75,-2.5);
\draw (7.25,2.5) to[out=280,in=90](7.5,0) to[out=270,in=80] (7.25,-2.5);
\draw (2.75,2.475) to[out=280,in=90](3,0) to[out=270,in=80] (2.75,-2.475);

%\draw (4.75,2) node {$\mathcal{P}_x$};
%\draw (4.95,1.5) node {$\cong G$};
\draw (9.5,-3.5) node {$M$};
\draw (5,2) node {$\mathcal{P}_U$};
\draw (3.5,-3.4) node {$($};
\filldraw [fill=red, draw=red] (4.25,-3.5) circle (1pt);
\draw (4.25,-3.9) node {\textcolor[rgb]{1,0,0}{$x$}};
\draw (5,-3.6) node {$)$};
\draw (4.25, -3) node {$U$};
\draw (0,0) node {$\mathcal{P}$};
\draw [->] (0,-0.4) -- (0,-3.1);
\draw (0.4,-1.75) node {$\pi$};
\draw (3.5,1) node {\textcolor[rgb]{1,0,0}{$p \cdot g$}};
%%%%%%%%%%%%%%%%%%%%%%%%%%%%%%%%%%%%%%%%%%%%%%%%%%%%%%%%%%%%%%%%%%%%%%%%%%%%%%%%%%%%
\draw (10.5,-3.5) to[out=10,in=170] (14.75,-3.5) to[out=350,in=190] (19,-3.5); %hori M
\filldraw [fill=gray!20!white, draw=white] (14,2.5) to[out=280,in=90](14.25,0) to[out=270,in=80] (14,-2.5) -- (15.5,-2.5)  to[out=80,in=270](15.75,0) to[out=90,in=280] (15.5,2.5) -- cycle; %gray area
\draw (11.75,2.5) to[out=280,in=90](12,0) to[out=270,in=80] (11.75,-2.5); %vert line 1
\draw (14.75,2.5) to[out=280,in=90](15,0) to[out=270,in=80] (14.75,-2.5); %3
\draw (16.25,2.5) to[out=280,in=90](16.5,0) to[out=270,in=80] (16.25,-2.5); %4
\draw (17.75,2.5) to[out=280,in=90](18,0) to[out=270,in=80] (17.75,-2.5); %5

\draw (13.25,2.475) to[out=280,in=90](13.5,0) to[out=270,in=80] (13.25,-2.475); %2
\filldraw [fill=blue, draw=blue] (15,0) circle (1pt);
\draw (15.5,0) node {\textcolor[rgb]{0,0,1}{$g$}};

\draw (19,0) node {$\mathcal{G}$}; %These three lines are for LGB projection
\draw [->] (19,-0.4) -- (19,-3.1);
%\draw (18.5,-1.75) node {$\pi_{\mathcal{G}}$};

\draw (15.5,2) node {$\mathcal{G}_U$};
\draw (14,-3.4) node {$($};
\filldraw [fill=red, draw=red] (14.75,-3.5) circle (1pt);
\draw (14.75,-3.9) node {\textcolor[rgb]{1,0,0}{$x$}};
\draw (15.5,-3.6) node {$)$};
\draw (14.75, -3) node {$U$};

\path[<-] (8.5,0) edge [bend left] (11,0);
\end{tikzpicture}
\end{figure}
\pause
But:
\bas
&&&r_g: \mathcal{P}_x \to \mathcal{P}_x\\
&\Rightarrow&&\mathrm{D}_pr_g \text{ only defined on vertical structure}
\eas
\end{frame}

\begin{frame}{Connection on $\mathcal{P}$: Idea}
\begin{figure}
%\caption{Pushforwards via right-multiplication of tangent vectors}  
%\label{figure:fibre bundle part two}
\centering
\begin{tikzpicture}[scale=0.55]
\draw (0,-3.5) to[out=10,in=170] (4.25,-3.5) to[out=350,in=190] (8.5,-3.5);
\draw (1.25,2.5) to[out=280,in=90](1.5,0) to[out=270,in=80] (1.25,-2.5);
\filldraw [fill=gray!20!white, draw=white] (3.5,2.5) to[out=280,in=90](3.75,0) to[out=270,in=80] (3.5,-2.5) -- (5,-2.5)  to[out=80,in=270](5.25,0) to[out=90,in=280] (5,2.5) -- cycle;
\draw (4.25,2.5) to[out=280,in=95](4.41,1) to[out=275,in=85] (4.41,-1) to[out=265,in=80] (4.25,-2.5); %draw with middle section for precise point placement

\draw (4,-1) node {\textcolor[rgb]{1,0,0}{$p$}};
\draw [<-, draw=blue] (5,1) to[out=275,in=85] (5,-1);
\draw [<-, draw=blue] (4.8,1) to[out=275,in=85] (4.8,-1);
\draw [<-, draw=blue] (4.6,1) to[out=275,in=85] (4.6,-1);
\draw (5.5,0) node {\textcolor[rgb]{0,0,1}{$\cdot \sigma$}};
\filldraw [fill=red, draw=red] (4.41,1) circle (1pt);
\filldraw [fill=red, draw=red] (4.41,-1) circle (1pt);
\draw (5.75,2.5) to[out=280,in=90](6,0) to[out=270,in=80] (5.75,-2.5);
\draw (7.25,2.5) to[out=280,in=90](7.5,0) to[out=270,in=80] (7.25,-2.5);
\draw (2.75,2.475) to[out=280,in=90](3,0) to[out=270,in=80] (2.75,-2.475);

%\draw (4.75,2) node {$\mathcal{P}_x$};
%\draw (4.95,1.5) node {$\cong G$};
\draw (9.5,-3.5) node {$M$};
\draw (5,2) node {$\mathcal{P}_U$};
\draw (3.5,-3.4) node {$($};
\filldraw [fill=red, draw=red] (4.25,-3.5) circle (1pt);
\draw (4.25,-3.9) node {\textcolor[rgb]{1,0,0}{$x$}};
\draw (5,-3.6) node {$)$};
\draw (4.25, -3) node {$U$};
\draw (0,0) node {$\mathcal{P}$};
\draw [->] (0,-0.4) -- (0,-3.1);
\draw (0.4,-1.75) node {$\pi$};
\draw (3.5,1) node {\textcolor[rgb]{1,0,0}{$p \cdot \sigma_x$}};
%%%%%%%%%%%%%%%%%%%%%%%%%%%%%%%%%%%%%%%%%%%%%%%%%%%%%%%%%%%%%%%%%%%%%%%%%%%%%%%%
\draw (10.5,-3.5) to[out=10,in=170] (14.75,-3.5) to[out=350,in=190] (19,-3.5); %hori M
\filldraw [fill=gray!20!white, draw=white] (14,2.5) to[out=280,in=90](14.25,0) to[out=270,in=80] (14,-2.5) -- (15.5,-2.5)  to[out=80,in=270](15.75,0) to[out=90,in=280] (15.5,2.5) -- cycle; %gray area
\draw (11.75,2.5) to[out=280,in=90](12,0) to[out=270,in=80] (11.75,-2.5); %vert line 1
\draw (14.75,2.5) to[out=280,in=90](15,0) to[out=270,in=80] (14.75,-2.5); %3
\draw (16.25,2.5) to[out=280,in=90](16.5,0) to[out=270,in=80] (16.25,-2.5); %4
\draw (17.75,2.5) to[out=280,in=90](18,0) to[out=270,in=80] (17.75,-2.5); %5

\draw (13.25,2.475) to[out=280,in=90](13.5,0) to[out=270,in=80] (13.25,-2.475); %2
\draw [draw=blue] (14.25,0) to[out=10,in=170] (15,0) to[out=350,in=190] (15.75,0);
%\filldraw [fill=blue, draw=blue] (15,0) circle (1pt);
\draw (13.9,0) node {\textcolor[rgb]{0,0,1}{$\sigma$}};

\draw (19,0) node {$\mathcal{G}$}; %These three lines are for LGB projection
\draw [->] (19,-0.4) -- (19,-3.1);
%\draw (18.5,-1.75) node {$\pi_{\mathcal{G}}$};

\draw (15.5,2) node {$\mathcal{G}_U$};
\draw (14,-3.4) node {$($};
\filldraw [fill=red, draw=red] (14.75,-3.5) circle (1pt);
\draw (14.75,-3.9) node {\textcolor[rgb]{1,0,0}{$x$}};
\draw (15.5,-3.6) node {$)$};
\draw (14.75, -3) node {$U$};

\path[<-] (8.5,0) edge [bend left] (11,0);
\end{tikzpicture}
\end{figure}

\bas
\text{Use } \sigma \in \Gamma(\mathcal{G}): r_\sigma(p) \coloneqq p \cdot \sigma_{x}
\eas
\end{frame}

\begin{frame}{Connection on $\mathcal{P}$: Revisiting the classical setup}
\setbeamercovered{invisible}
	\begin{minipage}[]{0.45\textwidth} 
	If $\mathcal{P}$ a typical principal bundle ($\mathcal{G}$ trivial, $\sigma \equiv g$ constant), and \textcolor[rgb]{0,0.58,0}{$H$} a connection:
	\begin{figure}
	\begin{tikzpicture}[scale=1.2]
		%Gray area
		\filldraw [fill=gray!20!white, draw=white] (3.5,2.5) to[out=280,in=95](3.66,1) to[out=275,in=85] (3.66,-1) to[out=265,in=80] (3.5,-2.5) -- (5,-2.5) to[out=80, in=265] (5.16, -1) to[out=85, in=275] (5.16, 1) to[out=95,in=280] (5,2.5) -- cycle;
		%lines
		\draw (4.25,2.5) to[out=280,in=95](4.41,1) to[out=275,in=85] (4.41,-1) to[out=265,in=80] (4.25,-2.5);
		\definecolor{darkgreen}{rgb}{0,0.58,0}
		\draw [draw=darkgreen] (3.66,-1) to[out=280,in=200] (4.41,-1) to[out=20,in=100] (5.16, -1);
		\draw [draw=darkgreen] (3.66,1) to[out=280,in=200] (4.41,1) to[out=20,in=100] (5.16,1);
		%Auxiliary stuff like arrows and dots
		\filldraw [fill=red, draw=red] (4.41,1) circle (1pt);
		\filldraw [fill=red, draw=red] (4.41,-1) circle (1pt);
		\draw [<-, draw=blue] (5.16,1) to[out=275,in=85] (5.16,-1);
		\draw [->, thick] (4.41,-1) -- (4.71,-0.88);
		%\draw [->] (4.41,1) -- (4.63,2);
		\draw [->, thick] (4.41,1) -- (4.71,1.12);
		%\draw [<-,dotted,thick] (4.63,2) -- (4.71,1.12);
		%Labels
		\draw (3.2,1) node {\textcolor[rgb]{0,0.58,0}{$H_{p \cdot g}$}};
		\draw (3.2,-1) node {\textcolor[rgb]{0,0.58,0}{$H_p$}};
		\draw (5.5,0) node {\textcolor[rgb]{0,0,1}{$\cdot g$}};
		\draw (3,2.2) node {$\mathcal{P}_U$};
		\draw (4.71,-0.6) node {$X$};
		\draw (5,1.4) node {$\mathrm{D}_pr_g(X)$};
		%\draw (5.5,1.56) node {$\thicksim \mathrm{D}_x\sigma(X)$};
		%Text
	\end{tikzpicture}
	\end{figure}
	\end{minipage}\hfill
	\pause
	\begin{minipage}[]{0.45\textwidth} 
	\begin{remark}[Integrated case]
	Parallel transport $\mathup{PT}^{\mathcal{P}}_\alpha$ in $\mathcal{P}$:
		\bas
		%\mathrm{D}_p r_g (X)
		%&=
		%\mleft.\frac{\mathrm{d}}{\mathrm{d}t}\mright|_{t=0}\mleft( \alpha \cdot g \mright),
		\mathup{PT}^{\mathcal{P}}_\alpha(p \cdot g)
		&=
		\mathup{PT}^{\mathcal{P}}_\alpha(p) \cdot g
		%\\
		%&=
		%\mathup{PT}^{\mathcal{P}}_\alpha(p) \cdot \mathup{PT}^{\mathcal{G}}_\alpha (g)
		\eas
		where $\alpha:I \to M$ is a base path 
	\end{remark}
	\end{minipage}
\end{frame}

\begin{frame}{Connection on $\mathcal{P}$: General case}
%\setbeamercovered{invisible}
	%\begin{minipage}[]{0.45\textwidth} 
	%%If $\mathcal{P}$ a typical principal bundle and \textcolor[rgb]{0,0.58,0}{$H$} a connection on it:
	%\begin{figure}
	%\begin{tikzpicture}[scale=1.2]
		%%Gray area
		%\filldraw [fill=gray!20!white, draw=white] (3.5,2.5) to[out=280,in=95](3.66,1) to[out=275,in=85] (3.66,-1) to[out=265,in=80] (3.5,-2.5) -- (5,-2.5) to[out=80, in=265] (5.16, -1) to[out=85, in=275] (5.16, 1) to[out=95,in=280] (5,2.5) -- cycle;
		%%lines
		%\draw (4.25,2.5) to[out=280,in=95](4.41,1) to[out=275,in=85] (4.41,-1) to[out=265,in=80] (4.25,-2.5);
		%\definecolor{darkgreen}{rgb}{0,0.58,0}
		%\draw [draw=darkgreen] (3.66,-1) to[out=280,in=200] (4.41,-1) to[out=20,in=100] (5.16, -1);
		%\draw [draw=darkgreen] (3.66,1) to[out=280,in=200] (4.41,1) to[out=20,in=100] (5.16,1);
		%%Auxiliary stuff like arrows and dots
		%\filldraw [fill=red, draw=red] (4.41,1) circle (1pt);
		%\filldraw [fill=red, draw=red] (4.41,-1) circle (1pt);
		%\draw [<-, draw=blue] (5.16,1) to[out=275,in=85] (5.16,-1);
		%\draw [->, thick] (4.41,-1) -- (4.71,-0.88);
		%\draw [->, thick] (4.41,1) -- (4.63,2);
		%\draw [->,dotted,thick] (4.41,1) -- (4.71,1.12);
		%\draw [<-,dotted,thick] (4.63,2) -- (4.71,1.12);
		%%Labels
		%\draw (3.2,1) node {\textcolor[rgb]{0,0.58,0}{$H_{p \cdot \sigma_x}$}};
		%\draw (3.2,-1) node {\textcolor[rgb]{0,0.58,0}{$H_p$}};
		%\draw (5.5,0) node {\textcolor[rgb]{0,0,1}{$\cdot \sigma$}};
		%\draw (3,2.2) node {$\mathcal{P}_U$};
		%\draw (4.71,-0.6) node {$X$};
		%\draw (5,2.25) node {$\mathrm{D}_pr_\sigma(X)$};
		%\draw (5.5,1.56) node {$\thicksim \mathrm{D}_p\sigma_\pi(X)$};
		%%Text
	%\end{tikzpicture}
	%\end{figure}
	%\end{minipage}\hfill
	%\begin{minipage}[]{0.5\textwidth} 
	%%Properties regarding the shift:
	%Now:
	%\bas
	%\mathrm{D}_p r_g(X) + \dots
		%&=
		%\mleft.\frac{\mathrm{d}}{\mathrm{d}t}\mright|_{t=0}\mleft( \alpha \cdot \sigma_{\pi \circ \alpha} \mright)
	%\eas
	%\pause
	%\begin{itemize}
		%\item Only vertical part in $\mathcal{G}$ of $\mathrm{D}\sigma$ matters
		%\item Shift is vertical in $\mathcal{P}$
	%\end{itemize}
	\begin{remark}[Integrated case]
	Ansatz: Introduce connection on $\mathcal{G}$,
		\bas
		\mathup{PT}^{\mathcal{P}}_\alpha(p \cdot g)
		&=
		\mathup{PT}^{\mathcal{P}}_\alpha(p) \cdot \mathup{PT}^{\mathcal{G}}_\alpha (g).
		\eas
	\end{remark}
	\pause
\begin{BackToTheRoots}[Part 2 of 3]
\begin{enumerate}
	\item $\mathcal{G} \cong M \times G$
	\item Equip $\mathcal{G}$ with canonical flat connection
\end{enumerate}
\end{BackToTheRoots}
\end{frame}
}

\section{Connection}
\subsection{Basic notions}

\renewcommand\insertreferences{{\tiny  Mark JD Hamilton. Mathematical Gauge Theory. \newline \textit{Springer}, 2017.}}

\begin{frame}{Classical situation: Differential of Lie group action}
\begin{remark}[Lie group $G$ situation with Lie algebra $\mathfrak{g}$]
In the case of a right $G$-action on $\mathcal{P}$, $\Phi: \mathcal{P} \times G\to \mathcal{P}$, we have
\bas
\mathrm{D}_{(p, g)}\Phi(X, Y)
&=
\mathrm{D}_pr_g(X)
	+ \mleft.\overline{(\mu_G)_g(Y)}\mright|_{p \cdot g}
\eas
for all $p \in \mathcal{P}$, $g \in G$, $X \in \mathrm{T}_p\mathcal{P}$ and $Y \in \mathrm{T}_g G$, where 
\begin{itemize}
	\item $\overline{\nu}$ denotes the fundamental vector field on $\mathcal{P}$ of $\nu \in \mathfrak{g}$,
	\item $\mu_G$ is the Maurer-Cartan form of G.
\end{itemize}
\end{remark}
\end{frame}
\renewcommand\insertreferences{{\tiny Straightforward generalization of classical definition as in: Mark JD Hamilton. Mathematical Gauge Theory. \newline \textit{Springer}, 2017.}}

\begin{frame}
\begin{definition}[Fundamental vector fields]
\textbf{Fundamental vector fields} defined by
\bas
\overline{\nu}_p
\coloneqq
\mleft.\frac{\mathrm{d}}{\mathrm{d}t}\mright|_{t = 0}\mleft( 
	p \cdot \mathup{e}^{t \nu_{x}}
\mright)
\eas
for all $\nu \in \Gamma(\mathcal{g})$ and $p \in \mathcal{P}_x$, where $\mathcal{g}$ is the LAB of $\mathcal{G}$.
\end{definition}
\end{frame}

\renewcommand\insertreferences{{\tiny Trivial generalization of classical definition as in: K. Mackenzie. General Theory of Lie Groupoids and Algebroids. \newline \textit{London Mathematical Society Lecture Note Series}, 213, 2005.}}

\begin{frame}
\setbeamercovered{invisible}
\begin{definition}[Darboux derivative]
For $\sigma \in \Gamma(\mathcal{G})$ we define the \textbf{Darboux derivative $\Delta \sigma \in \Omega^1(M; \mathcal{g})$}
\bas
\Delta \sigma
&=
\sigma^! \mu_{\mathcal{G}},
\eas
where $\mu_{\mathcal{G}}$ is given by
\bas
\mleft(\mu_{\mathcal{G}}\mright)_g
&\coloneqq
\mathrm{D}_gL_{g^{-1}} \circ \pi^v,
\eas
$\pi^v$ the projection onto the vertical bundle.
\end{definition}
\pause
\begin{remark}
If $\mathcal{G}$ a trivial LGB with canonical flat connection and if Lie group additionally a matrix group, then
\bas
\Delta\sigma
&=
\sigma^{-1} \mathrm{d}\sigma.
\eas
\end{remark}
\end{frame}

{
\setbeamertemplate{footline}{}
\setbeamercovered{invisible}
\begin{frame}
\begin{proposition}[Differential of LGB action $\Phi$, {[S.-R.\ F.]}]
We have
\bas
\mathrm{D}_{(p, g)}\Phi(X, Y)
&=
\mathrm{D}_pr_\sigma(X)
	- \mleft.{\overline{
		\mleft.\mleft(\pi^!\Delta \sigma\mright)\mright|_p (X)
	}}\mright|_{p \cdot g}
	+ \mleft.{\overline{
		\mleft( \mu_{\mathcal{G}}\mright)_{g} (Y)
	}}\mright|_{p \cdot g}
\eas
for all $(p, g) \in \mathcal{P}_x \times \mathcal{G}_x$, $(X, Y) \in \mathrm{T}_{(p, g)}(\mathcal{P} * \mathcal{G})$, where $\sigma$ is any section of $\mathcal{G}$ with $\sigma_{x} = g$.
\end{proposition}
\pause
\begin{definition}[Modified right-pushforward, {[S.-R.\ F.]}]
Define
\bas
\mathcal{r}_{g*}(X)
&\coloneqq
\mathrm{D}_pr_\sigma\mleft( X \mright)
	- \mleft.{\overline{
		\mleft. \mleft( \pi^!\Delta\sigma \mright) \mright|_p(X)
	}}\mright|_{p \cdot g}.
%\\
%\mathcal{r}_{\sigma*}(X)
%&\coloneqq
%\mathcal{r}_{\sigma_x*}(X).
\eas
\end{definition}
\end{frame}

\begin{frame}
\begin{proposition}[Well-defined isomorphism, {[S.-R.\ F.]}]
We have that
\bas
\mleft.\mathrm{T}\mathcal{P}\mright|_{\mathcal{P}_x} &\to \mleft.\mathrm{T}\mathcal{P}\mright|_{\mathcal{P}_x},\\
X 
&\mapsto 
\mathcal{r}_{g*}(X),
\eas
is a well-defined automorphism over $r_g$. 
%Similarly,
%\bas
%\mathrm{T}\mathcal{P} &\to \mathrm{T}\mathcal{P},\\
%X 
%&\mapsto 
%\mathcal{r}_{\sigma*}(X),
%\eas
%is an automorphism over $r_\sigma$.
\end{proposition}
\end{frame}
}

\subsection{Definitions}
{
\setbeamertemplate{footline}{}
\setbeamercovered{invisible}
\begin{frame}
\begin{definition}[Ehresmann connection, {[S.-R.\ F.]}]
$H$ a horizontal  distribution of $\mathrm{T}\mathcal{P}$ with
\bas
\mathcal{r}_{g*}\mleft( H_p \mright)
&=
H_{p\cdot g}
\eas
\end{definition}

\pause

\begin{definition}[Connection 1-form, {[S.-R.\ F.]}]
$A \in \Omega^1(\mathcal{P}; \pi^*\mathcal{g})$ with
\bas
A\mleft(\overline{\nu}\mright)
&=
\pi^*\nu,
\\
\mathcal{r}_\sigma^! A
&=
\mathrm{Ad}_{\sigma^{-1}} \circ A
\eas
for all $\sigma \in \Gamma(\mathcal{G})$ and $\nu \in \Gamma(\mathcal{g})$.
\end{definition}

%\pause

\begin{remark}
\bas
\mleft(\mathcal{r}_\sigma^! A\mright)_p(X)
&=
A_{p\sigma_x}\bigl(\mathcal{r}_{\sigma_x*}(X)\bigr).
\eas
\end{remark}
\end{frame}

\begin{frame}
\begin{theorem}[Equivalence of both definitions, {[S.-R.\ F.]}]
There is the usual 1:1 correspondence between both definitions:
\begin{itemize}
	\item Given $H$, define $A$ by
	\bas
	A_p(\overline{\nu}_p + X)
	&\coloneqq
	\mleft(\pi^*\nu\mright)_p,
	\eas
	where $X \in H_p$.
	\item Given $A$, define $H$ by
	\bas
	H_p
	&\coloneqq
	\mathrm{Ker}(A_p).
	\eas
\end{itemize}
\end{theorem}
\end{frame}

\subsection{Gauge transformation}
\begin{frame}
\begin{theorem}[Gauge transformation, {[S.-R.\ F.]}]
Let $s_i$, $s_j$ be two sections of $\mathcal{P}$ over $U_i$ and $U_j$, respectively, which are open subsets of $M$ with $U_i \cap U_j \neq \emptyset$. Then over $U_i \cap U_j$
\bas
A_{s_i}
&=
\mathrm{Ad}_{\sigma_{ji}^{-1}}\circ A_{s_j}
	+ \Delta\sigma_{ji},
\eas
where $A_{s_i} \coloneqq s_i^!A$ and $\sigma_{ji}$ a section of $\mathcal{G}$ with $s_i = s_j \cdot \sigma_{ji}$.
\end{theorem}
\end{frame}
}

\section{Curvature}
\subsection{Compatibility conditions}

\renewcommand\insertreferences{{\tiny Similar construction as in: Camille Laurent-Gengoux, Mathieu Stiénon, and Ping Xu. Non-abelian differentiable gerbes. \newline \textit{Advances in Mathematics}, 220(5):1357–1427, 2009.}}

\begin{frame}
\begin{proposition}[Connection on $\mathcal{g}$, {[S.-R.\ F.]}]
We have an induced vector bundle connection on $\mathcal{g}$ given by
\bas
\nabla^{\mathcal{G}} \nu
&\coloneqq
\mleft.\frac{\mathrm{d}}{\mathrm{d}t}\mright|_{t=0} \Delta \mathup{e}^{t \nu}.
\eas
\end{proposition}
\end{frame}

{
\setbeamertemplate{footline}{}

\begin{frame}
\begin{remark}
Recall, $\mathcal{G}$ a principal $\mathcal{G}$-bundle.
\end{remark}

\begin{definition}[Compatibility conditions, {[S.-R.\ F.]}]
$\mu_{\mathcal{G}}$ a \textbf{Yang-Mills connection (w.r.t.\ $\zeta \in \Omega^2(M; \mathcal{g})$)} if it satisfies the \textbf{compatibility conditions}:
\begin{enumerate}
	\item $\mu_{\mathcal{G}}$ a connection 1-form on $\mathcal{G} \stackrel{\pi_{\mathcal{G}}}{\to} M$;
	\item $\mu_{\mathcal{G}}$ satisfies the \textbf{generalised Maurer-Cartan equation}
	\bas
	\mleft.\mleft(\mathrm{d}^{\pi_{\mathcal{G}}^*\nabla^{\mathcal{G}}} \mu_{\mathcal{G}}
	+ \frac{1}{2} \mleft[ \mu_{\mathcal{G}} \stackrel{\wedge}{,} \mu_{\mathcal{G}} \mright]_{\pi_{\mathcal{G}}^*\mathcal{g}} \mright)\mright|_g
&=
\mathrm{Ad}_{g^{-1}} \circ \mleft.\pi_{\mathcal{G}}^!\zeta\mright|_g
	- \mleft.\pi_{\mathcal{G}}^!\zeta\mright|_g
	\eas
\end{enumerate}
\end{definition}
\end{frame}
}

\renewcommand\insertreferences{{\tiny For the first statement: Camille Laurent-Gengoux, Mathieu Stiénon, and Ping Xu. Non-abelian differentiable gerbes. \newline \textit{Advances in Mathematics}, 220(5):1357–1427, 2009.}}

\begin{frame}
\setbeamercovered{invisible}

\begin{proposition}[$\nabla^{\mathcal{G}}$ a Lie bracket derivation]
Let $\mu_{\mathcal{G}}$ be a connection 1-form on $\mathcal{G}$, then
\bas
\nabla^{\mathcal{G}}\mleft( \mleft[ \mu, \nu \mright]_{\mathcal{g}} \mright)
&=
\mleft[ \nabla^{\mathcal{G}} \mu, \nu \mright]_{\mathcal{g}}
	+ \mleft[ \mu, \nabla^{\mathcal{G}} \nu \mright]_{\mathcal{g}}.
\eas
\end{proposition}

\pause

\begin{theorem}[Curvature of LAB connection exact, {[S.-R.\ F.]}]
$\mu_{\mathcal{G}}$ satisfies the generalized Maurer-Cartan equation w.r.t.\ $\zeta$ if and only if 
\bas
R_{\nabla^{\mathcal{G}}}
&=
\mathrm{ad} \circ \zeta.
\eas
\end{theorem}
\end{frame}

\renewcommand\insertreferences{{\tiny For differential: Marius Crainic, Maria Amelia Salazar, and Ivan Struchiner. Multiplicative forms and Spencer operators. \newline \textit{Mathematische Zeitschrift}, 279(3):939–979, 2015.}}

\begin{frame}
%\setbeamercovered{invisible}
\begin{remark}
There is a simplicial differential $\delta$ on $\mathcal{G} \stackrel{\pi_{\mathcal{G}}}{\to} M$
\bas
\delta: \Omega^\bullet( \underbrace{\mathcal{G} * \dotsc * \mathcal{G}}_{k \text{ times}}; \pi_{\mathcal{G}}^*\mathcal{g} )
&\to
\Omega^\bullet( \underbrace{\mathcal{G} * \dotsc * \mathcal{G}}_{k + 1 \text{ times}}; \pi_{\mathcal{G}}^*\mathcal{g} )
\eas
such that the compatibility conditions are equivalent to
\bas
\delta \mu_{\mathcal{G}}
&=
0
&\text{ and }&&
\mathrm{d}^{\pi_{\mathcal{G}}^*\nabla^{\mathcal{G}}} \mu_{\mathcal{G}}
	+ \frac{1}{2} \mleft[ \mu_{\mathcal{G}} \stackrel{\wedge}{,} \mu_{\mathcal{G}} \mright]_{\pi_{\mathcal{G}}^*\mathcal{g}}
&=
\delta \zeta.
\eas
\end{remark}
%\end{frame}
\pause
%{
%\setbeamertemplate{footline}{}
%\begin{frame}{Guide to recover the classical theory: Part 3 of 3}
\begin{BackToTheRoots}[Part 3 of 3]
\begin{enumerate}
	\item $\mathcal{G} \cong M \times G$
	\item Equip $\mathcal{G}$ with canonical flat connection
	%\pause
	\item $\zeta \equiv 0$
\end{enumerate}
\end{BackToTheRoots}
\end{frame}
%}

\subsection{Definition and properties}

{
\setbeamertemplate{footline}{}
\begin{frame}
Given a Yang-Mills connection on $\mathcal{G}$:

\begin{definition}[Generalized curvature/field strength $F$ of $A$, {[S.-R.\ F.]}]
$\pi^H$ denotes the projection onto $H \subset \mathrm{T}\mathcal{P}$, then we define
\bas
F
&\coloneqq
\mathrm{d}^{\pi^*\nabla^{\mathcal{G}}} A \circ \mleft( \pi^{\mathrm{H}}, \pi^{\mathrm{H}} \mright)
	+ \pi^!\zeta.
\eas
\end{definition}

\begin{theorem}[Structure equation, {[S.-R.\ F.]}]
\bas
F
=
\mathrm{d}^{\pi^*\nabla^{\mathcal{G}}} A
	+ \frac{1}{2} \mleft[ A \stackrel{\wedge}{,} A \mright]_{\pi^*\mathcal{g}}
	+ \pi^!\zeta.
\eas
\end{theorem}
\end{frame}

\begin{frame}
\setbeamercovered{invisible}

\begin{proposition}[Properties of $F$, {[S.-R.\ F.]}]
\begin{itemize}
	\item $F(X, \cdot) = 0$, if $X$ vertical,
	\item $\mathcal{r}_{\sigma}^!F 	= 	\mathrm{Ad}_{\sigma^{-1}} \circ F$.
\end{itemize}
\end{proposition}

\pause

\begin{theorem}[Gauge transformation, {[S.-R.\ F.]}]
Let $s_i$, $s_j$ be two sections of $\mathcal{P}$ over $U_i$ and $U_j$, respectively, which are open subsets of $M$ with $U_i \cap U_j \neq \emptyset$. Then over $U_i \cap U_j$
\bas
F_{s_i}
&=
\mathrm{Ad}_{\sigma_{ji}^{-1}}\circ F_{s_j},
\eas
where $F_{s_i} \coloneqq s_i^!F$ and $\sigma_{ji}$ a section of $\mathcal{G}$ with $s_i = s_j \cdot \sigma_{ji}$.
\end{theorem}
\end{frame}
}

\section{Curved Yang-Mills gauge theory}
\subsection{Definition}

{
\setbeamertemplate{footline}{}

\begin{frame}
\begin{theorem}[Lagrangian, {[S.-R.\ F.]}]
\begin{itemize}
	\item $\kappa$ be an $\mathrm{Ad}$-invariant fibre metric on $\mathcal{g}$, 
	\item $M$ a spacetime, and $*$ its Hodge star operator, 
	\item $\mleft( U_i \mright)_i$ open covering of $M$ with subordinate gauges $s_i \in \Gamma\mleft(\mathcal{P}|_{U_i}\mright)$.
\end{itemize}
Then the Lagrangian $\mathfrak{L}_{\mathrm{CYM}}[A]$, defined locally by
\bas
\mleft.\bigl(\mathfrak{L}_{\mathrm{CYM}}[A]\bigr)\mright|_{U_i}
&\coloneqq 
- \frac{1}{2} \kappa \mleft( F_{s_i} \stackrel{\wedge}{,} *F_{s_i} \mright),
\eas
is well-defined, and
\bas
\mathfrak{L}_{\mathrm{CYM}}\mleft[ L^!A \mright]
&=
\mathfrak{L}_{\mathrm{CYM}}[A]
\eas
for all principal bundle automorphisms $L$.
\end{theorem}
\end{frame}

\subsection{Example}

\begin{frame}
\begin{example}[Hopf fibration $\mathds{S}^7 \to \mathds{S}^4$, {[S.-R.\ F.]}]
Let $P$ be the Hopf bundle
\begin{center}
	\begin{tikzcd}[ampersand replacement=\&, column sep=small]
		\mathrm{SU}(2) \cong \mathds{S}^3 \arrow{r}	\& \mathds{S}^7 \arrow{d} \\
			\& \mathds{S}^4
	\end{tikzcd}
\end{center}
Define $\mathcal{P} \coloneqq \mathcal{G}$ as the inner group bundle of $P$,
\bas
\mathcal{G}
&\coloneqq
c_{\mathrm{SU}(2)}(P)
\coloneqq 
\bigl(P\times \mathrm{SU}(2)\bigr) \Big/ \mathrm{SU}(2).
\eas
This principal $c_{\mathrm{SU}(2)}(P)$-bundle admits the structure as curved Yang-Mills gauge theory; there is no description as flat theory.
\end{example}
\end{frame}

}

\section{Conclusion}
{
\setbeamertemplate{footline}{}
%\begin{frame}{Summary}
%\begin{table}[h!]
		%\begin{tabularx}{\textwidth}{X| c c} 
			%\rowcolor{gray}
			%& Locally & Globally \\ \hline
			%Curved Yang-Mills & Pre-classical & $\mathrm{ad}(\mathds{S}^7 \to \mathds{S}^4)$ curved 
			%%\\
			%%\rowcolor{Gray}
			%%Tangent bundles & Pre-classical & $\mathrm{T}\mathds{S}^7$ curved \\
		%\end{tabularx}
%\end{table}
%\begin{remark}[Integrated point of view]
%This is probably linked to that an LGB is locally trivial 
%
%$\rightsquigarrow$ LGB action locally equivalent to a Lie group action
%\end{remark}
%
%\end{frame}

\begin{frame}
%{Hope: Structural Lie groupoids}
%\begin{table}[h!]
		%\begin{tabularx}{\textwidth}{X| c c} 
			%\rowcolor{gray}
			%Gauge theory & Minimal coupling evolves around... \\ \hline
			%Yang-Mills & $\mathrm{GL}(V) \curvearrowright V$ \\
			%& $V$ vector space \\
			%\rowcolor{Gray}
			%Curved Yang-Mills & $\mathrm{Aut}_{\mathrm{base}}(V) \curvearrowright V$ \\ 
			%\rowcolor{Gray}
			%& $V$ vector bundle, $\mathrm{Aut}_{\mathrm{base}}$ base-preserving \\
			%Curved Yang-Mills-Higgs & $\mathrm{Aut}_{\neg\mathrm{base}}(V) \curvearrowright V$ \\
			%& $V$ vector bundle, $\mathrm{Aut}_{\mathrm{base}}$ base-preserving
		%\end{tabularx}
%\end{table}
\begin{table}[h!]
		\begin{tabularx}{\textwidth}{X| c c} 
			\rowcolor{gray}
			Gauge theory & Structure \\ \hline
			Yang-Mills & Lie group $G$ \\
			\rowcolor{Gray}
			Curved Yang-Mills & Lie group bundle $\mathcal{G}$ \\ 
			Curved Yang-Mills-Higgs & Lie groupoid $\mathfrak{G}$?
		\end{tabularx}
\end{table}

\begin{center}
	\begin{tikzcd}[ampersand replacement=\&, column sep=small]
		\& \arrow{ld} \mathfrak{G} \arrow{rd} \& \\
		M \arrow{rr}{\Psi} \& \& N
	\end{tikzcd}
\end{center}

\begin{remark}
\begin{itemize}
	\item Richer set of principal bundles, containing Lie groupoids equipped with "non-flat Maurer-Cartan forms".
	\item Principal bundle for the whole of Yang-Mills-Higgs theory
	\item Even if $\mathcal{G}$ is trivial, what happens if its connection is not flat?
	\item May result into obstruction statements for curved Yang-Mills-Higgs gauge theories.
\end{itemize}
\end{remark}

\end{frame}
}
\thispagestyle{empty}
\topmargin -3.46 cm
\vspace*{\fill}
\begin{center}
\huge \textbf{Thank you!}
\end{center}
\vspace*{\fill}

\end{document}

